\section*{Глоссарий}
\begin{enumerate}
  \item REST (Representational State Transfer) -- архитектурный стиль взаимодействия компонентов распределённого приложения в сети. 
  
  \item Фронтенд -- серверное приложение, принимающее запросы от пользователя. На каждый из типов запросов определяется, как организовать его выполнение. Принимает запросы, анализирует их и в соответствии с заложенным алгоритмом выполняет запросы к бекендам.
  
  \item <<Горячее>> переконфигурирование системы -- способность системы применять изменения без перезапуска и перекомпиляции.
  
  \item Медиана времени отклика -- среднее время предоставления данных пользователю.
  
  \item Узел системы -- региональный сервер, содержащий данные авторов и читателей указанного региона.
  
  \item Валидация данных  -- проверка данных на соответствие заданным условиям и ограничениям.
  
  \item ПО -- программное обеспечение.
  
  \item Бекенд -- серверное приложение, выполняющее определенную задачу, например, взаимодействие с СУБД. Бекенд принимает запросы от фронтенда.

  \item API (Application Programming Interface) -- набор готовых методов и средств, позволяющих взаимодействовать с программным обеспечением.
  
  \item Хеширование данных -- процесс преобразования входных данных в хеш-значение фиксированной длины, которое представляет собой уникальное представление исходных данных.
\end{enumerate}