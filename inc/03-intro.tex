\section*{Введение}
По данным <<Исследования рынка туристических услуг в России>> от агентства MarketInsight, в 2022 году количество туристов, посетивших гостиничные учреждения, составило 64.8 миллиона человек, что на 67\% превысило показатель 2020 года (38.7 миллиона человек). Рост спроса на гостиничные услуги обусловлен увеличением внутреннего туризма и снижением международных поездок в связи с глобальной ситуацией.

Кроме того, количество гостиничных учреждений к концу 2022 года достигло 23.5 тысячи, в то время как в 2020 году их было 21.2 тысячи. Этот рост связан с развитием инфраструктуры и увеличением спроса на гостиничные услуги в стране.

В условиях острой конкуренции на рынке гостиничных услуг каждая компания стремится повысить качество сервиса, уделяя особое внимание развитию современных систем бронирования. Это позволяет компаниям улучшить удобство услуг для клиентов и обеспечить им возможность быстрого и эффективного бронирования номеров в гостиницах.

Таким образом, данные исследования подтверждают увеличение спроса на гостиничные услуги внутри страны, рост количества гостиничных учреждений и активное внедрение современных технологий для улучшения обслуживания клиентов и укрепления позиций на рынке гостиничных услуг.

Данное техническое задание составлено для разработки распределённой системы для  бронирования номеров апартаментов сети EASYGUEST.RU. Техническое задание выполнено на основе ГОСТ 19.201--78 <<ЕСПД. Техническое задание. Требования к содержанию и оформлению>>.